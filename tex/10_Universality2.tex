\documentclass{beamer}
%\usepackage{xspace}
\usepackage{amsmath,amssymb}
\usepackage{graphicx}
%\usepackage{svg}
%\usepackage{pgfpages}
%\pgfpagesuselayout{4 on 1}[a4paper,border shrink=5mm,landscape]
%\usepackage{psfrag}
%\usepackage[usenames,dvipsnames]{xcolor}
\usepackage{braket}
\usepackage{qcircuit}
\usepackage{tikz}
\usetikzlibrary{circuits.logic.US}
\usetikzlibrary{graphs}
\usetikzlibrary{datavisualization}
\usetikzlibrary{datavisualization.formats.functions}
\usepackage{pgfplotstable}
\usepgfplotslibrary{patchplots}

\setbeamercovered{transparent}

\usetheme{Pittsburgh}
%\usetheme{default}

\setbeamertemplate{sidebar right}{}
\setbeamertemplate{footline}[frame number]
%\usefonttheme{professionalfonts}

%\usepackage{sansmathaccent}
%\usepackage{bm}

%\usepackage{unicode-math}
%%\setmainfont[SlantedFont={Latin Modern Roman Slanted},SlantedFeatures={Color=000000},
%%  SmallCapsFont={TeX Gyre Termes},SmallCapsFeatures={Letters=SmallCaps}]{XITS}
%\setmathfont[math-style=ISO,sans-style=upright]{XITS Math}
%\setmathfont[range={\mathcal,\mathbfcal}]{Latin Modern Math}

\usepackage{sfmath}

%\mathversion{sans}

\newcommand{\Tr}{\mathsf{Tr}}

\definecolor{redorange}{rgb}{1.0, .25, .25}
\definecolor{citation}{rgb}{.1, 0.8, .35}
\newcommand\emm[1]{\textcolor{redorange}{{#1}}}
\newcommand\numc[1]{\textcolor{citation}{{\bf #1}}}

%\newcommand\bm[1]{{\mbox{\boldmath $#1$}}}
\newcommand\bm[1]{{\mathbf{#1}}}
%\newcommand\bm[1]{{\bf #1}}
%\newcommand\bm[1]{\ensuremath{\boldsymbol{#1}}}
%\newcommand\bm[1]{{\textbf{\it #1}}}

\title{Universality of quantum circuit}
\author{Ryuhei Mori}
%\institute{$\vcenter{\hbox{\includegraphics[width=30pt]{ELC_logo}}}$ Postdoctoral Fellow of ELC\\ $\vcenter{\hbox{\includegraphics[width=20pt]{titech_logo}}}$ Tokyo Institute of Technology}
\institute{Tokyo Institute of Technology}
%\date{21, Feb, 2019}



\begin{document}
\begin{frame}[plain]
\maketitle
\end{frame}



\begin{frame}{Whole quantum circuit}
Let $x = \mathtt{00\emm{1}01},\, y = \mathtt{11\emm{0}00}$.
\[
\Qcircuit @C=2em @R=2em {
 & \targ & \qw &\qw & \ctrl{1} & \qw & \qw & \targ &\qw \\
 & \qw & \targ &\qw & \ctrl{1} &\qw & \targ & \qw &\qw \\
 & \ctrl{-2} & \ctrl{-1} &\ctrl{2} & \gate{U} &\ctrl{2} & \ctrl{-1} & \ctrl{-2} & \qw \\
 & \qw & \qw &\qw & \ctrlo{-1} &\qw & \qw & \qw & \qw \\
 & \qw & \qw &\targ & \ctrlo{-1} &\targ & \qw & \qw & \qw
}
\]
\begin{align*}
\ket{00101}&\longmapsto \ket{11\emm{1}00}\longmapsto \ket{00101}\\
\ket{11000}&\longmapsto \ket{11\emm{0}00}\longmapsto \ket{11000}
\end{align*}
\end{frame}

\begin{frame}{Universality of a quantum circuit}
\begin{theorem}[Universality of finite gate set]
For any unitary matrix $U\in L(\mathbb{C}^{2^n})$ and $\emm{\epsilon} >0$,
there is a quantum circuit with \emm{$X,\,Y,\,Z,\,H,\,S,\,T,\,\mathsf{CNOT}$} gates computing $\widetilde{U}$
satisfying $\|U-\widetilde{U}\|<\emm{\epsilon}$.
\end{theorem}
\begin{proof}
\begin{enumerate}
\setlength{\itemsep}{1em}
\item Any unitary matrix can be decomposed to a product of \emm{two-level unitary matrices}. {\color{green}{Done}}
\item Any two-level unitary matrix can be decomposed to a product of \emm{controlled-unitary gates}. {\color{green}{Done}}
\item Any controlled-untary gate can be decomposed to a product of \emm{CNOT and arbitrary single-qubit gates}.
\item Any single-qubit gate can be approximated by $X,\,Y,\,Z,\,H,\,S$ and $T$.
\end{enumerate}
\end{proof}
\end{frame}

\begin{frame}{Controlled-unitary}
\begin{theorem}
Any controlled-unitary gate can be decomposed to a product of \emm{CNOT and arbitrary single-qubit gates}.
\end{theorem}
\begin{proof}
\begin{enumerate}
\setlength{\itemsep}{2em}
\item Controlled-unitary with \emm{single controlled qubit}.
\item Controlled-unitary with \emm{two controlled qubit}.
\item Controlled-unitary with \emm{$n$ controlled qubit}.
\end{enumerate}
\end{proof}
\end{frame}

\begin{frame}{Decomposition of single qubit unitary}
\begin{lemma}
Any single qubit unitary $U$, there is single qubit unitary matrices $A,\,B,\,C$ such that
$\emm{ABC=I}$ and $\emm{AXBXC=U}$.
\end{lemma}

\vspace{2em}
From this lemma,
\[
\Qcircuit @C=2em @R=2em {
& \ctrl{1} & \qw\\
& \gate{U} & \qw\\
}
\quad = \quad
\Qcircuit @C=2em @R=2em {
& \qw      & \ctrl{1} & \qw      & \ctrl{1} & \qw & \qw\\
& \gate{C} & \targ    & \gate{B} & \targ    & \gate{A} & \qw\\
}
\]
\end{frame}

\begin{frame}{Decomposition of single qubit unitary}
\begin{lemma}
Any single qubit unitary $U$, there is single qubit unitary matrices $A,\,B,\,C$ such that
$\emm{ABC=I}$ and $\emm{AXBXC=U}$.
\end{lemma}
\begin{proof}
\end{proof}
\end{frame}

\begin{frame}{Controlled-unitary}
\begin{theorem}
Any controlled-unitary gate can be decomposed to a product of \emm{CNOT and arbitrary single-qubit gates}.
\end{theorem}
\begin{proof}
\begin{enumerate}
\setlength{\itemsep}{2em}
\item Controlled-unitary with \emm{single controlled qubit}. {\color{green}{Done}}
\item Controlled-unitary with \emm{two controlled qubit}.
\item Controlled-unitary with \emm{$n$ controlled qubit}.
\end{enumerate}
\end{proof}
\end{frame}

\begin{frame}{Decomposition of two qubit unitary}
For $V$ satisfying $V^2=U$,

\[
\Qcircuit @C=2em @R=2em {
& \ctrl{1} & \qw\\
& \ctrl{1} & \qw\\
& \gate{U} & \qw\\
}
\quad = \quad
\Qcircuit @C=2em @R=2em {
& \ctrl{2} & \ctrl{1} & \qw              & \ctrl{1} & \qw      & \qw\\
& \qw      & \targ    & \ctrl{1}         & \targ    & \ctrl{1} & \qw\\
& \gate{V} & \qw      & \gate{V^\dagger} & \qw      & \gate{V} & \qw\\
}
\]
\end{frame}

\begin{frame}{Controlled-unitary}
\begin{theorem}
Any controlled-unitary gate can be decomposed to a product of \emm{CNOT and arbitrary single-qubit gates}.
\end{theorem}
\begin{proof}
\begin{enumerate}
\setlength{\itemsep}{2em}
\item Controlled-unitary with \emm{single controlled qubit}. {\color{green}{Done}}
\item Controlled-unitary with \emm{two controlled qubit}. {\color{green}{Done}}
\item Controlled-unitary with \emm{$n$ controlled qubit}.
\end{enumerate}
\end{proof}
\end{frame}

\begin{frame}{Controlled-unitary}
For $V$ satisfying $V^4=U$,
\begin{align*}
&\Qcircuit @C=2em @R=2em {
& \ctrl{1} & \qw\\
& \ctrl{1} & \qw\\
& \ctrl{1} & \qw\\
& \gate{U} & \qw\\
}
\quad =\\
&\Qcircuit @C=1em @R=2em {
& \ctrl{3} & \ctrl{1} & \qw              & \ctrl{1} & \qw      & \qw      & \qw              & \ctrl{2} & \qw      & \qw      & \qw              & \ctrl{2} & \qw      & \qw\\
& \qw      & \targ    & \ctrl{2}         & \targ    & \ctrl{2} & \ctrl{1} & \qw              & \qw      & \qw      & \ctrl{1} & \qw              & \qw      & \qw      & \qw\\
& \qw      & \qw      & \qw              & \qw      & \qw      & \targ    & \ctrl{1}         & \targ    & \ctrl{1} & \targ    & \ctrl{1}         & \targ    & \ctrl{1} & \qw\\
& \gate{V} & \qw      & \gate{V^\dagger} & \qw      & \gate{V} & \qw      & \gate{V^\dagger} & \qw      & \gate{V} & \qw      & \gate{V^\dagger} & \qw      & \gate{V} & \qw\\
}
\end{align*}
\end{frame}

\begin{frame}{Grey code}
\begin{equation*}
\mathtt{000}\mapsto
\mathtt{00\emm{1}}\mapsto
\mathtt{0\emm{1}1}\mapsto
\mathtt{01\emm{0}}\mapsto
\mathtt{\emm{1}10}\mapsto
\mathtt{11\emm{1}}\mapsto
\mathtt{1\emm{0}1}\mapsto
\mathtt{10\emm{0}}
\end{equation*}
\begin{equation*}
1\mapsto
2\mapsto
1\mapsto
3\mapsto
1\mapsto
2\mapsto
1
\end{equation*}

\vspace{2em}
\begin{equation*}
x_1 + x_2 + x_3 - (x_1\oplus x_2) - (x_2 \oplus x_3) - (x_3 \oplus x_1) + (x_1\oplus x_2\oplus x_3)
= 4(x_1\wedge x_2\wedge x_3)
\end{equation*}
\end{frame}

\begin{frame}{$n$ controlled qubits}
\begin{theorem}
For any single-qubit unitary $U$,
\end{theorem}
\begin{equation*}
\sum_{S\subseteq\{1,2,\dotsc,n\}} (-1)^{|S|+1}\left(\bigoplus_{i\in S} x_i\right) = 2^{n-1} \bigwedge_{i=1}^n x_i
\end{equation*}
\end{frame}

\begin{frame}{Universality of a quantum circuit}
\begin{theorem}[Universality of finite gate set]
For any unitary matrix $U\in L(\mathbb{C}^{2^n})$ and $\emm{\epsilon} >0$,
there is a quantum circuit with \emm{$X,\,Y,\,Z,\,H,\,S,\,T,\,\mathsf{CNOT}$} gates computing $\widetilde{U}$
satisfying $\|U-\widetilde{U}\|<\emm{\epsilon}$.
\end{theorem}
\begin{proof}
\begin{enumerate}
\setlength{\itemsep}{1em}
\item Any unitary matrix can be decomposed to a product of \emm{two-level unitary matrices}. {\color{green}{Done}}
\item Any two-level unitary matrix can be decomposed to a product of \emm{controlled-unitary gates}. {\color{green}{Done}}
\item Any controlled-untary gate can be decomposed to a product of \emm{CNOT and arbitrary single-qubit gates}. {\color{green}{Done}}
\item Any single-qubit gate can be approximated by $X,\,Y,\,Z,\,H,\,S$ and $T$.
\end{enumerate}
\end{proof}
\end{frame}

\begin{frame}{Universality of $X,Y,Z,H,S,T$}
\end{frame}

\begin{frame}{Assignments (Deadline is Jan.\ 17)}
\small
\begin{enumerate}
\setlength{\itemsep}{1em}
\item Show a decomposition of
\begin{equation*}
\frac12\begin{bmatrix}
1&1&1&1\\
1&i&-1&-i\\
1&-1&1&-1\\
1&-i&-1&i
\end{bmatrix}
\end{equation*}
into a product of two-level unitary matrices.
\item Show a decomposition of two-level unitary
\begin{equation*}
\begin{bmatrix}
1&0&0&0&0&0&0&0\\
0&1&0&0&0&0&0&0\\
0&0&a&0&0&0&0&c\\
0&0&0&1&0&0&0&0\\
0&0&0&0&1&0&0&0\\
0&0&0&0&0&1&0&0\\
0&0&0&0&0&0&1&0\\
0&0&b&0&0&0&0&d\\
\end{bmatrix}
\end{equation*}
into a product of controlled-unitary gates.
\end{enumerate}
\end{frame}


\end{document}
