\documentclass{beamer}
%\usepackage{xspace}
\usepackage{amsmath,amssymb}
\usepackage{graphicx}
%\usepackage{svg}
%\usepackage{pgfpages}
%\pgfpagesuselayout{4 on 1}[a4paper,border shrink=5mm,landscape]
%\usepackage{psfrag}
%\usepackage[usenames,dvipsnames]{xcolor}
\usepackage{blkarray}
\usepackage{braket}
\usepackage{qcircuit}
\usepackage{tikz}
\usepackage{tikz-3dplot}
%\usetikzlibrary{tikz-3dplot}
\usetikzlibrary{graphs}
\usetikzlibrary{datavisualization}
\usetikzlibrary{datavisualization.formats.functions}
\usepackage{pgfplotstable}
\usepgfplotslibrary{patchplots}

\setbeamercovered{transparent}

\usetheme{Pittsburgh}
%\usetheme{default}

\setbeamertemplate{sidebar right}{}
\setbeamertemplate{footline}[frame number]
%\usefonttheme{professionalfonts}

%\usepackage{sansmathaccent}
%\usepackage{bm}

%\usepackage{unicode-math}
%%\setmainfont[SlantedFont={Latin Modern Roman Slanted},SlantedFeatures={Color=000000},
%%  SmallCapsFont={TeX Gyre Termes},SmallCapsFeatures={Letters=SmallCaps}]{XITS}
%\setmathfont[math-style=ISO,sans-style=upright]{XITS Math}
%\setmathfont[range={\mathcal,\mathbfcal}]{Latin Modern Math}

\usepackage{sfmath}

%\mathversion{sans}

\newcommand{\Tr}{\mathsf{Tr}}

\definecolor{redorange}{rgb}{1.0, .25, .25}
\definecolor{citation}{rgb}{.1, 0.8, .35}
\newcommand\emm[1]{\textcolor{redorange}{{#1}}}
\newcommand\numc[1]{\textcolor{citation}{{\bf #1}}}

%\newcommand\bm[1]{{\mbox{\boldmath $#1$}}}
\newcommand\bm[1]{{\mathbf{#1}}}
%\newcommand\bm[1]{{\bf #1}}
%\newcommand\bm[1]{\ensuremath{\boldsymbol{#1}}}
%\newcommand\bm[1]{{\textbf{\it #1}}}

\title{Quantum teleportation}
\author{Ryuhei Mori}
%\institute{$\vcenter{\hbox{\includegraphics[width=30pt]{ELC_logo}}}$ Postdoctoral Fellow of ELC\\ $\vcenter{\hbox{\includegraphics[width=20pt]{titech_logo}}}$ Tokyo Institute of Technology}
\institute{Tokyo Institute of Technology}
%\date{21, Feb, 2019}



\begin{document}
\begin{frame}[plain]
\maketitle
\end{frame}

\begin{frame}{Time evolution of a system}
Time evolution of a system is represented by a map from a state to a state.

\vspace{2em}
\begin{equation*}
T: \text{The set of states} \to \text{the set of states}.
\end{equation*}

\vspace{2em}
\begin{equation*}
p T(\rho_1) + (1-p) T(\rho_2) = T(p\rho_1 + (1-p)\rho_2)
\end{equation*}
for any density matrices $\rho_1,\,\rho_2$ and $p\in[0,1]$.

\vspace{2em}
\begin{center}
$T$ must be \emm{linear} (a proof is needed).
\end{center}
\end{frame}




\begin{frame}{Schr\"odinger picture and Heisenberg picture}
\begin{equation*}
\emm{T^\dagger}: \text{The set of binary measurements} \to \text{the set of binary measurements}.
\end{equation*}

\vspace{1em}
\begin{equation*}
\langle T(\rho), P\rangle
=
\langle \rho, \emm{T^\dagger}(P)\rangle
\end{equation*}
for any $\rho\in H(V)$ and $P\in H(W)$.
$\emm{T^\dagger}$ is an \emm{adjoint} map of $T$.
\begin{align*}
&\langle T_3(T_2(T_1(\rho))), P\rangle
=
\langle T_2(T_1(\rho)), T_3^\dagger(P)\rangle\\
=&
\langle T_1(\rho), T_2^\dagger(T_3^\dagger(P))\rangle
= \langle \rho, T_1^\dagger(T_2^\dagger(T_3^\dagger(P)))\rangle
\end{align*}
\end{frame}

\begin{frame}{No-cloning theorem}
\begin{align*}
\ket{0}\bra{0} &\longmapsto \ket{0}\bra{0} \otimes \ket{0}\bra{0} \\
\ket{1}\bra{1} &\longmapsto \ket{1}\bra{1} \otimes \ket{1}\bra{1}
\end{align*}
From the linearlity,
\begin{align*}
\frac12(\ket{0}\bra{0} + \ket{1}\bra{1}) &\longmapsto \frac12(\ket{0}\bra{0} \otimes \ket{0}\bra{0} + \ket{1}\bra{1} \otimes \ket{1}\bra{1}) \\
\end{align*}
This is not equal to
\begin{align*}
\frac12(\ket{0}\bra{0} + \ket{1}\bra{1})\otimes \frac12(\ket{0}\bra{0} + \ket{1}\bra{1}).
\end{align*}
\end{frame}

\begin{frame}{Axioms for operations}
$T\colon H(V) \to H(W)$.

\vspace{1em}
\begin{enumerate}
\setlength{\itemsep}{2em}
\item Trace preserving: $\Tr(T(\rho)) = \Tr(\rho)$.
\onslide<1>{\item Positive : $T(\rho)\succeq 0$ for any $\rho\succeq 0$.}
\onslide<2>{\item Completely positive: $\mathsf{id}\otimes T$ is positive.}
\end{enumerate}

%\vspace{3em}
%In the following, we only consider $T\colon \rho \longmapsto U\rho U^\dagger$.
\end{frame}

\begin{frame}{Unitary operations}
\begin{equation*}
\rho \longmapsto U\rho U^\dagger.
\end{equation*}
\begin{enumerate}
\setlength{\itemsep}{2em}
\item Trace preserving: $\Tr(U\rho U^\dagger) = \Tr(\rho)$.
\item Completely positive: 
$(\mathsf{id}\otimes T)(\rho) = (I\otimes U)\rho (I\otimes U^\dagger)$.
%$\mathsf{id}\otimes T$ is positive.
\end{enumerate}

\vspace{2em}
\begin{center}
In the most of quantum computing, only \emm{pure} states and \emm{unitary} operations are used.
\end{center}
\end{frame}

\begin{frame}{Controlled not}
\begin{center}
\mbox{
\Qcircuit @C=1em @R=2.0em {
  \lstick{\ket{x}} &  \ctrl{1} & \rstick{\ket{x}} \qw\\
  \lstick{\ket{y}} &  \targ    & \rstick{\ket{y\oplus x}} \qw
}
}
\end{center}
\begin{align*}
\mathsf{CNOT} \ket{x}\ket{y} \longmapsto \ket{x}\ket{y\oplus x}
\end{align*}
\begin{align*}
\mathsf{CNOT}=
\begin{bmatrix}
1&0&0&0\\
0&1&0&0\\
0&0&0&1\\
0&0&1&0
\end{bmatrix}.
\end{align*}
\end{frame}

\begin{frame}{Bell states and quantum circuit}

\vspace{1em}
\centering
\mbox{
\Qcircuit @C=1em @R=.7em {
  \lstick{\ket{0}} & \gate{H} & \ctrl{1} &  \qw & \qw \\
  \lstick{\ket{0}} & \qw      & \targ    &  \qw & \qw 
}
}

\vspace{2em}
\begin{align*}
\ket{0}\ket{0} &\longmapsto
\frac1{\sqrt{2}}(\ket{0}+\ket{1})\ket{0}
=
\frac1{\sqrt{2}}(\ket{0}\ket{0}+\ket{1}\ket{0})\\
 &\longmapsto
\frac1{\sqrt{2}}(\ket{0}\ket{0}+\ket{1}\ket{1})
\end{align*}

\vspace{1em}
\begin{align*}
\ket{x}\ket{y} &\longmapsto
\frac1{\sqrt{2}}(\ket{0}+(-1)^x\ket{1})\ket{y}
=
\frac1{\sqrt{2}}(\ket{0}\ket{y}+(-1)^x\ket{1}\ket{y})\\
 &\longmapsto
\frac1{\sqrt{2}}(\ket{0}\ket{y}+(-1)^x\ket{1}\ket{\bar{y}}).
\end{align*}

\end{frame}

\begin{frame}{Conditional probability}
A probability of outcome of local measurement in a joint system is
\begin{align*}
P(a, b) = \Tr(\rho (P_a \otimes Q_b)).
\end{align*}
\begin{align*}
P(a\mid b) &= \frac1{P(b)} \Tr(\rho (P_a \otimes Q_b))\\
 &=  \frac1{P(b)}\Tr(\emm{\Tr_W(\rho (I\otimes Q_b))} P_a).
\end{align*}
For $Q_b=\ket{\psi_b}\bra{\psi_b}$,
\begin{equation*}
\emm{\Tr_W(\rho (I\otimes Q_b))}
=
\Tr_W(\rho (I\otimes \ket{\psi_b}\bra{\psi_b}))
\end{equation*}
\end{frame}


\begin{frame}{Quantum teleportation}

\[
\Qcircuit @C=1.8em @R=0.8em @!R {
\lstick{\ket{\psi}}& \qw & \qw & \ctrl{1} & \gate{H} & \meter & \cctrl{2}\\
\lstick{\ket{0}} & \qw     & \targ & \targ & \qw & \meter \cwx[1] &\\
\lstick{\ket{0}} & \gate{H}     & \ctrl{-1}   & \qw & \qw   & \gate{X} & \gate{Z} & \rstick{\ket{\psi}} \qw
}
\]
\end{frame}

\begin{frame}{Assignments \small[Deadline is the next Friday]}
\begin{enumerate}
\setlength{\itemsep}{2em}
\item Show the density matrix $\rho\in H(\mathbb{C}^2)$ of the Bell state
\begin{equation*}
\frac1{\sqrt{2}}(\ket{00}+\ket{11}) \in \mathbb{C}^2\otimes \mathbb{C}^2
\end{equation*}
when $\ket{0}\bra{0}$ is measured at the second system.
\item Show the density matrix $\rho\in H(\mathbb{C}^2)$ of the Bell state
when $\ket{+}\bra{+}$ is measured at the second system.
\item Show the density matrix $\rho\in H(\mathbb{C}^2)$ of the Bell state
when $\ket{\psi}\bra{\psi}$ is measured at the second system where $\ket{\psi}:=\alpha\ket{0}+\beta\ket{1}$.
\end{enumerate}
\end{frame}

\end{document}
