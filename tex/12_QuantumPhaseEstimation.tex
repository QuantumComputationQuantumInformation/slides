\documentclass{beamer}
%\usepackage{xspace}
\usepackage{amsmath,amssymb}
\usepackage{graphicx}
%\usepackage{svg}
%\usepackage{pgfpages}
%\pgfpagesuselayout{4 on 1}[a4paper,border shrink=5mm,landscape]
%\usepackage{psfrag}
%\usepackage[usenames,dvipsnames]{xcolor}
\usepackage{braket}
\usepackage{qcircuit}
%\usepackage{algorithmicx}
\usepackage{algpseudocode}
\usepackage{tikz}
\usepackage{tikz-3dplot}
\usetikzlibrary{circuits.logic.US}
\usetikzlibrary{graphs}
\usetikzlibrary{datavisualization}
\usetikzlibrary{datavisualization.formats.functions}
\usepackage{pgfplotstable}
\usepgfplotslibrary{patchplots}

\setbeamercovered{transparent}

\usetheme{Pittsburgh}
%\usetheme{default}

\setbeamertemplate{sidebar right}{}
\setbeamertemplate{footline}[frame number]
%\usefonttheme{professionalfonts}

%\usepackage{sansmathaccent}
%\usepackage{bm}

%\usepackage{unicode-math}
%%\setmainfont[SlantedFont={Latin Modern Roman Slanted},SlantedFeatures={Color=000000},
%%  SmallCapsFont={TeX Gyre Termes},SmallCapsFeatures={Letters=SmallCaps}]{XITS}
%\setmathfont[math-style=ISO,sans-style=upright]{XITS Math}
%\setmathfont[range={\mathcal,\mathbfcal}]{Latin Modern Math}

\usepackage{sfmath}

%\mathversion{sans}

\newcommand{\Tr}{\mathsf{Tr}}

\definecolor{redorange}{rgb}{1.0, .25, .25}
\definecolor{citation}{rgb}{.1, 0.8, .35}
\newcommand\emm[1]{\textcolor{redorange}{{#1}}}
\newcommand\numc[1]{\textcolor{citation}{{\bf #1}}}

%\newcommand\bm[1]{{\mbox{\boldmath $#1$}}}
\newcommand\bm[1]{{\mathbf{#1}}}
%\newcommand\bm[1]{{\bf #1}}
%\newcommand\bm[1]{\ensuremath{\boldsymbol{#1}}}
%\newcommand\bm[1]{{\textbf{\it #1}}}

\title{Quantum phase estimation}
\author{Ryuhei Mori}
%\institute{$\vcenter{\hbox{\includegraphics[width=30pt]{ELC_logo}}}$ Postdoctoral Fellow of ELC\\ $\vcenter{\hbox{\includegraphics[width=20pt]{titech_logo}}}$ Tokyo Institute of Technology}
\institute{Tokyo Institute of Technology}
%\date{21, Feb, 2019}



\begin{document}
\begin{frame}[plain]
\maketitle
\end{frame}


\begin{frame}{Quantum algorithms}
\begin{itemize}
\setlength{\itemsep}{4em}
\item \emm{Quantum phase estimation}: Integer factoring.
\item Grover search, quantum walk: Unstructured search.
\end{itemize}
\end{frame}

\begin{frame}{Hadamard test}
\[
\Qcircuit @C=2em @R=1em {
\lstick{\ket{0}}   &\gate{H} & \ctrl{1} & \gate{H} & \meter\\
\lstick{\ket{\psi}} & {/} \qw & \gate{U} & \qw      & \qw
}
\]

\begin{align*}
\ket{0}\ket{\psi}&\longmapsto \frac{\ket{0}+\ket{1}}{\sqrt{2}}\ket{\psi} \longmapsto
\frac1{\sqrt{2}}\left(\ket{0}\ket{\psi}+\ket{1}U\ket{\psi}\right)\\
&\longmapsto
\frac1{\sqrt{2}}\left(\ket{+}\ket{\psi}+\ket{-}U\ket{\psi}\right)\\
&=\frac1{2}\left(\ket{0}(\ket{\psi}+U\ket{\psi})+\ket{1}(\ket{\psi}-U\ket{\psi})\right).
\end{align*}
0 is measured with probability $\left\|\frac{\ket{\psi}+U\ket{\psi}}{2}\right\|^2 = \frac{1+\emm{\mathsf{Re}(\bra{\psi}U\ket{\psi})}}2$.

1 is measured with probability $\left\|\frac{\ket{\psi}-U\ket{\psi}}{2}\right\|^2 = \frac{1-\emm{\mathsf{Re}(\bra{\psi}U\ket{\psi})}}2$.
\end{frame}

\begin{frame}{Hadamard test}
\[
\Qcircuit @C=2em @R=1em {
\lstick{\ket{0}}   &\gate{H} & \ctrl{1} & \gate{S} & \gate{H} & \meter\\
\lstick{\ket{\psi}} & {/} \qw      & \gate{U} & \qw & \qw      & \qw
}
\]

\begin{align*}
\ket{0}\ket{\psi}&\longmapsto \frac{\ket{0}+\ket{1}}{\sqrt{2}}\ket{\psi} \longmapsto
\frac1{\sqrt{2}}\left(\ket{0}\ket{\psi}+\emm{i}\ket{1}U\ket{\psi}\right)\\
&\longmapsto
\frac1{\sqrt{2}}\left(\ket{+}\ket{\psi}+\emm{i}\ket{-}U\ket{\psi}\right)\\
&=\frac1{2}\left(\ket{0}(\ket{\psi}+\emm{i}U\ket{\psi})+\ket{1}(\ket{\psi}-\emm{i}U\ket{\psi})\right).
\end{align*}
0 is measured with probability $\left\|\frac{\ket{\psi}+\emm{i}U\ket{\psi}}{2}\right\|^2 = \frac{1+\emm{\mathsf{Im}(\bra{\psi}U\ket{\psi})}}2$.

1 is measured with probability $\left\|\frac{\ket{\psi}-\emm{i}U\ket{\psi}}{2}\right\|^2 = \frac{1-\emm{\mathsf{Im}(\bra{\psi}U\ket{\psi})}}2$.
\end{frame}

\if0
\begin{frame}{The number of measurements}
\begin{equation*}
\Pr\left(\left|\frac{\sum_i X_i}n - \mathbb{E}[X]\right| > \frac{\epsilon}{\sqrt{n}}\right)
\end{equation*}
\end{frame}

\begin{frame}{Example: Swap test}
Let $\mathsf{SWAP}\ket{\psi}\ket{\varphi}=\ket{\varphi}\ket{\psi}$.
\[
\Qcircuit @C=2em @R=1em {
\lstick{\ket{0}}   &\gate{H} & \ctrl{1} & \gate{H} & \meter\\
\lstick{\ket{\psi}} &{/}\qw      & \multigate{1}{\mathsf{SWAP}} & \qw      & \qw\\
\lstick{\ket{\varphi}} &{/}\qw      & \ghost{\mathsf{SWAP}} & \qw      & \qw
}
\]

\vspace{2em}
Then, $\mathsf{Re}(\bra{\psi}\bra{\varphi}\mathsf{SWAP}\ket{\psi}\ket{\varphi}) =  |\emm{\braket{\psi|\varphi}}|^2$.
\end{frame}
\fi

%\begin{frame}{Problems}
%Eigenvector
%\end{frame}

\begin{frame}{Hadamard test for eigenvector}
If $\ket{\psi_\theta}$ is an eigenvector of $U$ for eigenvalue $\mathsf{e}^{i\theta}$.

\vspace{1em}
\begin{align*}
\mathsf{Re}(\bra{\psi_\theta}U\ket{\psi_\theta}) &= \mathsf{Re}(\mathsf{e}^{i\theta})
= \cos(\theta)\\
\mathsf{Im}(\bra{\psi_\theta}U\ket{\psi_\theta}) &= \mathsf{Im}(\mathsf{e}^{i\theta})
= \sin(\theta)
\end{align*}

\vspace{1em}
Hence, we can estimate $\theta$. But, this algorithm doesn't work when \emm{the eigenvector $\ket{\psi_\theta}$ is not given}.
For
\begin{align*}
\ket{\psi} := \sum_{j=1}^N \alpha_j \ket{\psi_{\theta_j}}
\end{align*}
$\bra{\psi}U\ket{\psi}=\sum_{j=1}^N |\alpha_j|^2 \mathsf{e}^{i\theta_j}$.

\end{frame}

\begin{frame}{Quantum phase estimation}
\[
\Qcircuit @C=2em @R=1em {
\lstick{\ket{0}}   &\gate{H} & \ctrl{3} & \qw&\qw &    \qw\\
\lstick{\ket{0}}   &\gate{H} & \qw &  \ctrl{2} &\qw &   \qw\\
\lstick{\ket{0}}   &\gate{H} & \qw &  \qw & \ctrl{1} &  \qw\\
\lstick{\ket{\psi_\theta}} &{/}\qw      & \gate{U} & \gate{U^2} &\gate{U^4} &  \qw
}
\]

\begin{align*}
\ket{0}^n\ket{\psi_\theta} &\longmapsto \frac1{\sqrt{2^n}}\sum_{x\in\{0,1\}^n} \ket{x}\ket{\psi_\theta}\\
&\longmapsto \frac1{\sqrt{2^n}}\sum_{x\in\{0,1\}^n} \mathsf{e}^{i\theta(x_1 + 2x_2 + 2^2 x_3 + \dotsm 2^{n-1} x_n)}\ket{x}\ket{\psi_\theta}\\
&= \frac1{\sqrt{2^n}}\sum_{x\in\emm{\{0,1,2,\dotsc,2^n-1\}}} \mathsf{e}^{i\theta x}\ket{x}\ket{\psi_\theta}\\
\end{align*}
\end{frame}

\begin{frame}{Quantum Fourier transform}
\begin{align*}
\begin{bmatrix}
1\\0\\0\\0
\end{bmatrix}
,
\begin{bmatrix}
0\\1\\0\\0
\end{bmatrix}
,
\begin{bmatrix}
0\\0\\1\\0
\end{bmatrix}
,
\begin{bmatrix}
0\\0\\0\\1
\end{bmatrix}
\longleftrightarrow
\frac1{2}
\begin{bmatrix}
1\\1\\1\\1
\end{bmatrix}
,
\frac1{2}
\begin{bmatrix}
1\\\omega\\\omega^2\\\omega^3
\end{bmatrix}
,
\frac1{2}
\begin{bmatrix}
1\\\omega^2\\\omega^{4}\\\omega^{6}
\end{bmatrix}
,
\frac1{2}
\begin{bmatrix}
1\\\omega^3\\\omega^{6}\\\omega^{9}
\end{bmatrix}
\end{align*}
\begin{align*}
\ket{x} \longleftrightarrow \emm{\ket{\widehat{x}}}:=\frac1{\sqrt{N}}\sum_{y=0}^{N-1} \omega_N^{xy} \ket{y}
\end{align*}
where $\omega_N:=\mathsf{e}^{i\frac{2\pi}{N}}$.
\begin{align*}
U_{\mathsf{QFT}} &:= \sum_{x=0}^{N-1}\emm{\ket{\widehat{x}}}\bra{x}
\end{align*}
Hadamard operator is the quantum Fourier transform for $N=2$.
\end{frame}

\begin{frame}{Quantum phase estimation}
\[
\Qcircuit @C=2em @R=1em {
\lstick{\ket{0}}   &\gate{H} & \ctrl{3} & \qw&\qw & \multigate{2}{\mathsf{IQFT}}& \rstick{\ket{\varphi_0}}  \qw\\
\lstick{\ket{0}}   &\gate{H} & \qw &  \ctrl{2} &\qw & \ghost{IQFT} & \rstick{\ket{\varphi_1}}\qw\\
\lstick{\ket{0}}   &\gate{H} & \qw &  \qw & \ctrl{1} & \ghost{IQFT}& \rstick{\ket{\varphi_2}}\qw\\
\lstick{\ket{\psi_\theta}} &{/}\qw      & \gate{U} & \gate{U^2} &\gate{U^4} &\qw & \rstick{\ket{\psi_\theta}}\qw
}
\]

\vspace{1em}
\emm{Assume} $\theta = 2\pi \frac{\varphi}{2^n}$ for some \emm{integer} $\varphi\in\{0,1,\dotsc,2^n-1\}$.
\begin{align*}
&\frac1{\sqrt{2^n}}\sum_{x\in\{0,1,2,\dotsc,2^n-1\}} \mathsf{e}^{i\frac{2\pi}{2^n}\varphi x}\ket{x}\ket{\psi_\theta}\\
=&\ket{\widehat{\varphi}}\ket{\psi_\theta}\\
\longmapsto&\ket{\varphi}\ket{\psi_\theta}
\end{align*}
\end{frame}

\begin{frame}{Probability of the best approximation}
\small
Assume that the eigenvalue is $\mathsf{e}^{2\pi i\phi}$.
Let $\varphi\in\{0,1,\dotsc,2^n-1\}$ be the best $n$-bit approximation of $\phi$, i.e., \emm{$|\phi - \frac{\varphi}{2^n}|\le \frac1{2^{n+1}}$}.
\begin{align*}
\Pr(\varphi) &= \frac1{2^{2n}}\left|\left(\sum_{x=0}^{2^n-1} \mathsf{e}^{-i\frac{2\pi}{2^n}\varphi x}\bra{x}\right)\left(\sum_{x=0}^{2^n-1} \mathsf{e}^{i 2\pi\phi x}\ket{x}\right)\right|^2\\
&= \frac1{2^{2n}}\left|\sum_{x=0}^{2^n-1} \mathsf{e}^{2\pi i\left(\phi - \frac{\varphi}{2^n}\right) x}\right|^2\\
&= \frac1{2^{2n}}\left|\frac{1-\mathsf{e}^{2\pi i\left(2^n\phi - \varphi\right)}}{1-\mathsf{e}^{2\pi i\left(\phi - \frac{\varphi}{2^n}\right)}}\right|^2\\
&= \frac1{2^{2n}}\frac{2\sin^2\left(\pi\left(2^n\phi - \varphi\right)\right)}{2\sin^2\left(\pi\left(\phi - \frac{\varphi}{2^n}\right)\right)}\\
&\ge \frac1{2^{2n}}\frac{\sin^2\left(\pi\left(2^n\phi - \varphi\right)\right)}{\left(\pi\left(\phi - \frac{\varphi}{2^n}\right)\right)^2}\\
&\ge \frac1{2^{2n}}\frac{\left(2\left(2^n\phi - \varphi\right)\right)^2}{\left(\pi\left(\phi - \frac{\varphi}{2^n}\right)\right)^2}
= \frac{4}{\pi^2}\approx 0.405
\end{align*}
\end{frame}

\if0
\begin{frame}{Error probability}
\begin{align*}
\Pr\left(\left|\phi - \frac{\varphi}{2^n}\right| > \frac{c}{2^n}\right)
&=\sum_{} \frac1{2^{2n}}\frac{\sin^2\left(\pi\left(2^n\phi - \varphi\right)\right)}{\sin^2\left(\pi\left(\phi - \frac{\varphi}{2^n}\right)\right)}\\
&= \frac1{2^{2n}}\sum_{}\frac{1}{\left(\pi\left(\phi - \frac{\varphi}{2^n}\right)\right)^2}\\
\end{align*}
\end{frame}
\fi

\begin{frame}{Quantum phase estimation for superposition of eigenvectors}
\begin{align*}
\ket{\psi} := \sum_{i=1}^N \alpha_i \ket{\psi_{\theta_i}}
\end{align*}
\begin{align*}
\ket{0}\left(\sum_{i=1}^N \alpha_i \ket{\psi_{\theta_i}}\right)
&\longmapsto \sum_{i=1}^N \alpha_i \ket{\widehat{\varphi}_i}\ket{\psi_{\theta_i}}\\
&\longmapsto \sum_{i=1}^N \alpha_i \ket{\varphi_i}\ket{\psi_{\theta_i}}
\end{align*}
Then, \emm{$\varphi_i$} is measured with probability $|\alpha_i|^2$.
\end{frame}

\begin{frame}{Quantum Fourier transform}
\small
\vspace{-1em}
%\begin{align*}
%\ket{\widehat{x}} := \frac1{\sqrt{N}} \sum_{z=0}^{N-1} \omega_N^{xz}\ket{z}
%\end{align*}
%Quantum Fourier transform
\begin{align*}
U_{\mathsf{QFT}(N)} := \sum_{x=0}^{N-1}\ket{\widehat{x}}\bra{x}.
\end{align*}
\begin{align*}
U_{\mathsf{QFT}(2^n)} &= \frac1{\sqrt{2^{n}}}\sum_{x=0}^{2^n-1}\sum_{z=0}^{2^n-1}\omega_{2^n}^{xz}\ket{z}\bra{x}\\
&= \frac1{\sqrt{2^{n}}}\sum_{x=0}^{2^n-1}\sum_{z\in\{0,1\}^n}\omega_{2^n}^{x(z_1+2z_2+\dotsb +2^{n-1}z_n)}\ket{z}\bra{x}\\
&= \frac1{\sqrt{2^{n}}}\sum_{x=0}^{2^n-1}\sum_{z\in\{0,1\}^n}\left(\bigotimes_{i=1}^n\omega_{2^n}^{x2^{i-1}z_i}\ket{z_i}\right)\bra{x}\\
&= \frac1{\sqrt{2^{n}}}\sum_{x=0}^{2^n-1}\bigotimes_{i=1}^n \left(\ket{0} + \omega_{2^n}^{x2^{i-1}}\ket{1}\right)\bra{x}\\
&= \sum_{x=0}^{2^n-1}\bigotimes_{i=1}^n \frac{\ket{0} + \omega_{2^{n-i+1}}^{x}\ket{1}}{\sqrt{2}}\bra{x}.
\end{align*}
\end{frame}

\begin{frame}{Quantum Fourier transform}
\small
\vspace{-1em}
\begin{align*}
&= \sum_{x=0}^{2^n-1}\bigotimes_{i=1}^n \frac{\ket{0} + \omega_{2^{n-i+1}}^{x}\ket{1}}{\sqrt{2}}\bra{x}\\
&= \sum_{x\in\{0,1\}^n}\bigotimes_{i=1}^n \frac{\ket{0} + \omega_{2^{n-i+1}}^{x_1+2x_2+\dotsb+2^{n-1}x_n}\ket{1}}{\sqrt{2}}\bra{x}\\
&= \sum_{x\in\{0,1\}^n}\bigotimes_{i=1}^n \frac{\ket{0} + \omega_{2^{n-i+1}}^{x_1+2x_2+\dotsb+2^{n-i}x_{n-i+1}}\ket{1}}{\sqrt{2}}\bra{x}
\end{align*}
Let $R_k:=\begin{bmatrix}1&0\\0&\omega_{2^k}\end{bmatrix}$.
\[
\hspace{-5em}
\Qcircuit @C=.5em @R=.8em {
\lstick{\ket{x_1}} & \qw      & \qw        & \qw        & \ctrl{3}   & \qw      & \qw        & \ctrl{2}   & \qw      & \ctrl{1}   & \gate{H} & \rstick{\frac{\ket{0}+\omega_{2^1}^{x_1}\ket{1}}{\sqrt{2}}} \qw\\
\lstick{\ket{x_2}} & \qw      & \qw        & \ctrl{2}   & \qw        & \qw      & \ctrl{1}   & \qw        & \gate{H} & \gate{R_2} & \qw      & \rstick{\frac{\ket{0}+\omega_{2^2}^{x_1+2x_2}\ket{1}}{\sqrt{2}}} \qw\\
\lstick{\ket{x_3}} & \qw      & \ctrl{1}   & \qw        & \qw        & \gate{H} & \gate{R_2} & \gate{R_3} & \qw      & \qw        & \qw      & \rstick{\frac{\ket{0}+\omega_{2^3}^{x_1+2x_2+2^2x_3}\ket{1}}{\sqrt{2}}} \qw\\
\lstick{\ket{x_4}} & \gate{H} & \gate{R_2} & \gate{R_3} & \gate{R_4} & \qw      & \qw        & \qw        & \qw      & \qw        & \qw      & \rstick{\frac{\ket{0}+\omega_{2^4}^{x_1+2x_2+2^2x_3+2^3x_4}\ket{1}}{\sqrt{2}}} \qw\\
}
\]
\end{frame}

\begin{frame}{Whole quantum circuit of QFT}
\small
Let $R_k:=\begin{bmatrix}1&0\\0&\omega_{2^k}\end{bmatrix}$.
\[
\hspace{-0em}
\Qcircuit @C=1.2em @R=.8em {
\lstick{\ket{x_1}} & \qw      & \qw        & \qw        & \ctrl{3}   & \qw      & \qw        & \ctrl{2}   & \qw      & \ctrl{1}   & \gate{H} & \qswap     & \qw    & \qw\\
\lstick{\ket{x_2}} & \qw      & \qw        & \ctrl{2}   & \qw        & \qw      & \ctrl{1}   & \qw        & \gate{H} & \gate{R_2} & \qw      & \qw\qwx    & \qswap    & \qw\\
\lstick{\ket{x_3}} & \qw      & \ctrl{1}   & \qw        & \qw        & \gate{H} & \gate{R_2} & \gate{R_3} & \qw      & \qw        & \qw      & \qw\qwx    & \qswap\qwx    & \qw\\
\lstick{\ket{x_4}} & \gate{H} & \gate{R_2} & \gate{R_3} & \gate{R_4} & \qw      & \qw        & \qw        & \qw      & \qw        & \qw      & \qswap\qwx & \qw    & \qw
}
\]
\end{frame}

\if0
\begin{frame}{Quantum Fourier transform}
\small
\vspace{-1em}
%\begin{align*}
%\ket{\widehat{x}} := \frac1{\sqrt{N}} \sum_{z=0}^{N-1} \omega_N^{xz}\ket{z}
%\end{align*}
%Quantum Fourier transform
\begin{align*}
U_{\mathsf{QFT}(N)} := \sum_{x=0}^{N-1}\ket{\widehat{x}}\bra{x}.
\end{align*}
%If $N=2^n$,
%\begin{align*}
%&U_{\mathsf{QFT}(2^n)} = \frac1{\sqrt{2^{n}}}\sum_{x=0}^{2^n-1}\sum_{z=0}^{2^n-1}\omega_{2^n}^{xz}\ket{z}\bra{x}\\
%&= \frac1{\sqrt{2^{n}}}\sum_{z=0}^{2^{n}-1}\ket{z}\sum_{x=0}^{2^{\emm{n-1}}-1}\left(\omega_{2^n}^{2xz}\bra{2x} + \omega_{2^n}^{(2x+1)z}\bra{2x+1}\right)\\
%&= \frac1{\sqrt{2^{n}}}\sum_{z=0}^{2^{n}-1}\ket{z}\sum_{x=0}^{2^{\emm{n-1}}-1}\left(\omega_{2^{\emm{n-1}}}^{xz}\bra{x}\bra{0} + \omega_{2^n}^z \omega_{2^{\emm{n-1}}}^{xz}\bra{x}\bra{1}\right)\\
%&= \frac1{\sqrt{2^{n}}}\sum_{z=0}^{2^{\emm{n-1}}-1}\Biggl[\ket{0}\ket{z}\sum_{x=0}^{2^{\emm{n-1}}-1}\left(\omega_{2^{\emm{n-1}}}^{xz}\bra{x}\bra{0} + \omega_{2^n}^z \omega_{2^{\emm{n-1}}}^{xz}\bra{x}\bra{1}\right)\\
%&\quad +\ket{1}\ket{z}\sum_{x=0}^{2^{\emm{n-1}}-1}\left(\omega_{2^{\emm{n-1}}}^{xz}\bra{x}\bra{0} - \omega_{2^n}^z \omega_{2^{\emm{n-1}}}^{xz}\bra{x}\bra{1}\right)\Biggr]\\
%%&= \frac1{\sqrt{2^{n}}}\sum_{x=0}^{2^n-1}\left(\sum_{z=0}^{2^{\emm{n-1}}-1}\omega_{2^n}^{xz}\ket{z} + \omega_{2^n}^{x(z+2^{n-1})}\ket{z+2^{n-1}}\right)\bra{x}\\
%%&= \frac1{\sqrt{2^{n}}}\sum_{x=0}^{2^n-1}\left(\sum_{z=0}^{2^{\emm{n-1}}-1}\omega_{2^n}^{xz}\ket{z} + (-1)^x\omega_{2^n}^{xz}\ket{z+2^{n-1}}\right)\bra{x}\\
%%&= \frac1{\sqrt{2^{n}}}\sum_{x=0}^{2^{\emm{n-1}}-1}\left(\sum_{z=0}^{2^{\emm{n-1}}-1}\omega_{2^n}^{xz}\ket{z} + (-1)^x\omega_{2^n}^{xz}\ket{z+2^{n-1}}\right)\bra{x}\\
%%&= U_{\mathsf{QFT}(2^{n-1})} R\otimes \ket{0}\bra{+} + U_{\mathsf{QFT}(2^{n-1})}R\left(\sum_{x=0}^{2^{n-1}-1} \omega_{2^n}^x\ket{x}\bra{x}\right)\otimes \ket{1}\bra{-}\\
%\end{align*}
\begin{align*}
&U_{\mathsf{QFT}(2^n)} = \frac1{\sqrt{2^{n}}}\sum_{x=0}^{2^n-1}\sum_{z=0}^{2^n-1}\omega_{2^n}^{xz}\ket{z}\bra{x}\\
&= \frac1{\sqrt{2^{n}}}\sum_{x=0}^{2^n-1}\left(\sum_{z=0}^{2^{\emm{n-1}}-1}\omega_{2^n}^{x2z}\ket{2z} + \omega_{2^n}^{x(2z+1)}\ket{2z+1}\right)\bra{x}\\
&= \frac1{\sqrt{2^{n}}}\sum_{x=0}^{2^n-1}\left(\sum_{z=0}^{2^{\emm{n-1}}-1}\omega_{2^{\emm{n-1}}}^{xz}\ket{z}\ket{0} + \omega_{2^n}^x\omega_{2^{n-1}}^{xz}\ket{z}\ket{1}\right)\bra{x}\\
%&= \frac1{\sqrt{2^{n}}}\sum_{x=0}^{2^{n-1}-1}\left(\sum_{z=0}^{2^{\emm{n-1}}-1}\omega_{2^{\emm{n-1}}}^{xz}\ket{z}\ket{0} + \omega_{2^n}^x\omega_{2^{n-1}}^{xz}\ket{z}\ket{1}\right)\bra{x}(\bra{0}+\bra{1})\\
%&= \frac1{\sqrt{2^{n-1}}}\sum_{x=0}^{2^{\emm{n-1}}-1}\left(\sum_{z=0}^{2^{\emm{n-1}}-1}\omega_{2^{\emm{n-1}}}^{xz}\ket{z}\ket{0}\bra{+}\bra{x} + \omega_{2^{n-1}}^x\omega_{2^{n-1}}^{xz}\ket{z}\ket{1}\bra{-}\bra{x}\right)\\
%&= U_{\mathsf{QFT}(2^{n-1})} R\otimes \ket{0}\bra{+} + U_{\mathsf{QFT}(2^{n-1})}R\left(\sum_{x=0}^{2^{n-1}-1} \omega_{2^n}^x\ket{x}\bra{x}\right)\otimes \ket{1}\bra{-}\\
&= \frac1{\sqrt{2^{n}}}\sum_{x=0}^{2^{\emm{n-1}}-1}\Biggl[\left(\sum_{z=0}^{2^{\emm{n-1}}-1}\omega_{2^{\emm{n-1}}}^{xz}\ket{z}\ket{0} + \omega_{2^n}^x\omega_{2^{n-1}}^{xz}\ket{z}\ket{1}\right)\bra{0}\bra{x}\\
&\qquad + \left(\sum_{z=0}^{2^{\emm{n-1}}-1}\omega_{2^{\emm{n-1}}}^{xz}\ket{z}\ket{0} - \omega_{2^n}^x\omega_{2^{n-1}}^{xz}\ket{z}\ket{1}\right)\bra{1}\bra{x}\Biggr]
\end{align*}
\end{frame}
\fi

\if0
\begin{frame}{Quantum Fourier transform}
\small
\begin{align*}
&= \frac1{\sqrt{2^{n}}}\sum_{z=0}^{2^{\emm{n-1}}-1}\Biggl[\ket{0}\ket{z}\sum_{x=0}^{2^{\emm{n-1}}-1}\left(\omega_{2^{\emm{n-1}}}^{xz}\bra{x}\bra{0} + \omega_{2^n}^z \omega_{2^{\emm{n-1}}}^{xz}\bra{x}\bra{1}\right)\\
&\quad +\ket{1}\ket{z}\sum_{x=0}^{2^{\emm{n-1}}-1}\left(\omega_{2^{\emm{n-1}}}^{xz}\bra{x}\bra{0} - \omega_{2^n}^z \omega_{2^{\emm{n-1}}}^{xz}\bra{x}\bra{1}\right)\Biggr]
\end{align*}
Let $S_n:=  \sum_{x\in\{0,1\}^n} \ket{x_1x_nx_{n-1}\dotsm x_2}\bra{x_nx_{n-1}\dotsm x_1}$.
\begin{align*}
&= \frac1{\sqrt{2^{n}}}S_n\sum_{z=0}^{2^{\emm{n-1}}-1}\Biggl[\ket{z}\ket{0}\sum_{x=0}^{2^{\emm{n-1}}-1}\left(\omega_{2^{\emm{n-1}}}^{xz}\bra{x}\bra{0} + \omega_{2^n}^z \omega_{2^{\emm{n-1}}}^{xz}\bra{x}\bra{1}\right)\\
&\quad +\ket{z}\ket{1}\sum_{x=0}^{2^{\emm{n-1}}-1}\left(\omega_{2^{\emm{n-1}}}^{xz}\bra{x}\bra{0} - \omega_{2^n}^z \omega_{2^{\emm{n-1}}}^{xz}\bra{x}\bra{1}\right)\Biggr]\\
&= S_n \left[U_{\mathsf{QFT}(2^{n-1})}\otimes \ket{+}\bra{0} + D_{2^n}U_{\mathsf{QFT}(2^{n-1})}\otimes \ket{-}\bra{1} \right]
\end{align*}
where $D_{2^{n-1}}:=\sum_{x=0}^{2^{n-1}-1} \omega_{2^{n}}^x\ket{x}\bra{x}$.
\end{frame}
\fi

\if0
\begin{frame}{Quantum Fourier transform}
\small
\begin{align*}
%&= \frac1{\sqrt{2^{n-1}}}\sum_{x=0}^{2^{\emm{n-1}}-1}\left(\sum_{z=0}^{2^{\emm{n-1}}-1}\omega_{2^{\emm{n-1}}}^{xz}\ket{z}\ket{0}\bra{+}\bra{x} + \omega_{2^{n-1}}^x\omega_{2^{n-1}}^{xz}\ket{z}\ket{1}\bra{-}\bra{x}\right)
%&= U_{\mathsf{QFT}(2^{n-1})} R\otimes \ket{0}\bra{+} + U_{\mathsf{QFT}(2^{n-1})}R\left(\sum_{x=0}^{2^{n-1}-1} \omega_{2^n}^x\ket{x}\bra{x}\right)\otimes \ket{1}\bra{-}\\
&= \frac1{\sqrt{2^{n}}}\sum_{x=0}^{2^{\emm{n-1}}-1}\Biggl[\left(\sum_{z=0}^{2^{\emm{n-1}}-1}\omega_{2^{\emm{n-1}}}^{xz}\ket{z}\ket{0} + \omega_{2^n}^x\omega_{2^{n-1}}^{xz}\ket{z}\ket{1}\right)\bra{0}\bra{x}\\
&\qquad + \left(\sum_{z=0}^{2^{\emm{n-1}}-1}\omega_{2^{\emm{n-1}}}^{xz}\ket{z}\ket{0} - \omega_{2^n}^x\omega_{2^{n-1}}^{xz}\ket{z}\ket{1}\right)\bra{1}\bra{x}\Biggr]
\end{align*}
Let $S_n:=  \sum_{x\in\{0,1\}^n} \ket{x_{n-1}x_{n-2}\dotsm x_1x_n}\bra{x_nx_{n-1}\dotsm x_1}$.
\begin{align*}
&= \frac1{\sqrt{2^{n}}}S_n\sum_{x=0}^{2^{\emm{n-1}}-1}\Biggl[\left(\sum_{z=0}^{2^{\emm{n-1}}-1}\omega_{2^{\emm{n-1}}}^{xz}\ket{0}\ket{z} + \omega_{2^n}^x\omega_{2^{n-1}}^{xz}\ket{1}\ket{z}\right)\bra{0}\bra{x}\\
&\qquad + \left(\sum_{z=0}^{2^{\emm{n-1}}-1}\omega_{2^{\emm{n-1}}}^{xz}\ket{0}\ket{z} - \omega_{2^n}^x\omega_{2^{n-1}}^{xz}\ket{1}\ket{z}\right)\bra{1}\bra{x}\Biggr]\\
&= S_n \left[\ket{0}\bra{+}\otimes U_{\mathsf{QFT}(2^{n-1})} + \ket{1}\bra{-}\otimes U_{\mathsf{QFT}(2^{n-1})}D_{2^{n-1}}\right]
\end{align*}
%\begin{align*}
%&= \frac1{\sqrt{2^{n-1}}}\sum_{x=0}^{2^{\emm{n-1}}-1}\left(\sum_{z=0}^{2^{\emm{n-1}}-1}\omega_{2^{\emm{n-1}}}^{xz}R_n\ket{0}\ket{z}\bra{+}\bra{x} + \omega_{2^{n-1}}^x\omega_{2^{n-1}}^{xz}R_n\ket{1}\ket{z}\bra{-}\bra{x}\right)\\
%&= R_n\left(\ket{0}\bra{+}\otimes U_{\mathsf{QFT}(2^{n-1})}\right) + R_n\left(\ket{1}\bra{-}\otimes U_{\mathsf{QFT}(2^{n-1})}D_{2^{n-1}}\right)
%\end{align*}
where $D_{2^{n-1}}:=\sum_{x=0}^{2^{n-1}-1} \omega_{2^{n}}^x\ket{x}\bra{x}$.
%\begin{align*}
%&= R_n\left(\ket{0}\bra{0}\otimes U_{\mathsf{QFT}(2^{n-1})}\right)(H\otimes I) + R_n\left(\ket{1}\bra{1}\otimes U_{\mathsf{QFT}(2^{n-1})}D_{2^{n-1}}\right)(H\otimes I)\\
%&= R_n\left(I\otimes U_{\mathsf{QFT}(2^{n-1})}\right)(C)(H\otimes I)
%\end{align*}
\end{frame}

\begin{frame}{Circuit of QFT}
\small
\begin{align*}
U_{\mathsf{QFT}(2^n)} &= S_n \left[\ket{0}\bra{+}\otimes U_{\mathsf{QFT}(2^{n-1})} + \ket{1}\bra{-}\otimes U_{\mathsf{QFT}(2^{n-1})}D_{2^{n-1}}\right]\\
&= S_n \left[\ket{0}\bra{0}\otimes U_{\mathsf{QFT}(2^{n-1})} + \ket{1}\bra{1}\otimes U_{\mathsf{QFT}(2^{n-1})}D_{2^{n-1}}\right][H\otimes I_{2^{n-1}}]\\
&= S_n [I_2\otimes U_{\mathsf{QFT}(2^{n-1})}]\left[\ket{0}\bra{0}\otimes I_{2^{n-1}} + \ket{1}\bra{1}\otimes D_{2^{n-1}}\right][H\otimes I_{2^{n-1}}]\\
\end{align*}
\[
\Qcircuit @C=2em @R=1em {
& \qw      & \multigate{2}{D_{2^{n-1}}} & \multigate{2}{U_{\mathsf{QFT}(2^{n-1})}} & \multigate{3}{S_n} &  \qw\\
& \qw      & \ghost{D_{2^{n-1}}}        & \ghost{U_{\mathsf{QFT}(2^{n-1})}}        & \ghost{S_n}        &  \qw\\
& \qw      & \ghost{D_{2^{n-1}}}        & \ghost{U_{\mathsf{QFT}(2^{n-1})}}        & \ghost{S_n}        &  \qw\\
& \gate{H} & \ctrl{-1}                  &                                      \qw & \ghost{S_n}        &  \qw\\
}
\]
\end{frame}

\begin{frame}{Circuit of QFT}
\[
\Qcircuit @C=1em @R=1em {
& \qw      & \multigate{2}{D_{2^{n-1}}} & \qw      &\multigate{1}{D_{2^{n-2}}} & \multigate{1}{U_{\mathsf{QFT}(2^{n-2})}} & \multigate{2}{S_{n-1}} & \multigate{3}{S_n} &  \qw\\
& \qw      & \ghost{D_{2^{n-1}}}        & \qw      &\ghost{D_{2^{n-2}}}        & \ghost{U_{\mathsf{QFT}(2^{n-2})}}        & \ghost{S_{n-1}}        & \ghost{S_n}        &  \qw\\
& \qw      & \ghost{D_{2^{n-1}}}        & \gate{H} &\ctrl{-1}                  & \qw                                      & \ghost{S_{n-1}}        & \ghost{S_n}        &  \qw\\
& \gate{H} & \ctrl{-1}                  & \qw      &                       \qw & \qw                                      & \qw                    & \ghost{S_n}        &  \qw\\
}
\]
\end{frame}
\fi

\if0
\begin{frame}{Quantum Fourier transform}
\begin{align*}
&\ket{\psi} := \sum_{x=0}^{2^n-1} \psi_x \ket{x} \longleftrightarrow \frac1{\sqrt{2^n}}\sum_{x=0}^{2^n-1} \psi_x \sum_{y=0}^{2^n-1} \omega_{2^n}^{xy} \ket{y}\\
=&\, \frac1{\sqrt{2^n}}\sum_{x=0}^{2^{n-1}-1} \psi_{2x} \sum_{y=0}^{2^n-1} \omega_{2^n}^{2xy} \ket{y}
+ \frac1{\sqrt{2^n}}\sum_{x=0}^{2^{n-1}-1} \psi_{2x+1} \sum_{y=0}^{2^n-1} \omega_{2^n}^{(2x+1)y} \ket{y}\\
=&\, \frac1{\sqrt{2^n}}\sum_{x=0}^{2^{n-1}-1} \psi_{2x} \sum_{y=0}^{2^n-1} \omega_{2^{n-1}}^{xy} \ket{y}
+ \frac{\omega_{2^n}^y}{\sqrt{2^n}}\sum_{x=0}^{2^{n-1}-1} \psi_{2x+1} \sum_{y=0}^{2^n-1} \omega_{2^{n-1}}^{xy} \ket{y}\\
=& \frac1{\sqrt{2}} U_{n-1}\ket{\psi_{\mathsf{even}}} + \frac{\omega_{2^n}^y}{\sqrt{2}} U_{n-1}\ket{\psi_{\mathsf{odd}}}
\end{align*}
\end{frame}
\fi

%\begin{frame}{Example: Period finding}
%\end{frame}

\if0
\begin{frame}{Controlled-unitary for $U(2)$}
\begin{theorem}
For any $U\in\mathsf{U}(2)$, controlled-$U$ gate with $n$ controlled qubits can be realized by $O(n^2)$ CNOT and arbitrary single-qubit gates without ancillas (working qubits).
\end{theorem}
\begin{proof}
\small
Induction on $n$. For $V\in U(2)$ satisfying $V^2=U$,
\[
\Qcircuit @C=2em @R=1em {
& \ctrl{6} & \qw      & \ctrl{6} & \qw     & \qw\\
& \ctrl{5} & \qw      & \ctrl{5} & \qw     & \qw\\
& \ctrl{4} & \qw      & \ctrl{4} & \qw     & \qw\\
& \ctrl{3} & \ctrl{3} & \qw      & \ctrl{3}& \qw\\
& \ctrl{2} & \ctrl{2} & \qw      & \ctrl{2}& \qw\\
& \qw      & \ctrl{1} & \qw      & \ctrl{1}& \qw\\
& \gate{V} & \gate{V^\dagger} & \gate{W} & \gate{V}& \qw
}
\]
$S_n = 4 S_{n/2} = 4^{\log n}S_1 = O(\emm{n^2})$.
\end{proof}
\end{frame}
\fi


\begin{frame}{Assignments (Deadline is Jan.\ 31)}
\small
\begin{enumerate}
\setlength{\itemsep}{2em}
\item Show that the Fourier basis $\{\ket{\widehat{x}}\}_{x\in\{0,1,\dotsc,N-1\}}$ is orthonormal.
\end{enumerate}
\end{frame}

\end{document}
